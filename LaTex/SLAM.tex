\documentclass[11pt]{article}
\usepackage{graphicx}
\usepackage[czech]{babel}
\usepackage[utf8]{inputenc}
\usepackage{titling}
\usepackage{pdfpages}
\usepackage[nopar]{lipsum}
\usepackage{mathtools}
\usepackage{multirow}
\usepackage{caption}
\usepackage{float}
\usepackage{enumitem}
\usepackage{listings}
\usepackage{amsmath}
\usepackage{amssymb}
\setlength{\parindent}{1cm}
\setlength{\hoffset}{-1.1cm} 
\setlength{\voffset}{-2cm}
\setlength{\textheight}{23.0cm} 
\setlength{\textwidth}{15cm}

\begin{document}
\title{Projekt 4}
\author{Lukáš Kuhajda}
\date{Akademický rok 2017/2018}
\begin{titlepage}
	\begin{center}
		\includegraphics[scale=0.5]{logo_zcu}\\
		\vspace{5cm}
		\begin{Large}
			\textbf{\thetitle}\\
		\end{Large}
		
		\vspace{3cm}
		\theauthor\\
		\vspace{5cm}
		\thedate
	\end{center}
\end{titlepage}
\newpage	
	
\tableofcontents
\newpage

\section{Simultaneous localization and mapping: part I.}
\subsection{Úvod}
Simultaneous localization and mapping, v překladu simultánní lokalizace a mapování (SLAM) je problém, při kterém řešíme otázku, zda-li dokáže robot, umístěný do neznámého prostředí a do neznámé lokace, postupně tvořit mapu svého okolí, zatímco určuje, kde se právě nachází. Řešení tohoto problému je nezbytně nutné pro tvorbu zcela autonomně se pohybujících robotů. \\
\subsection{Historie}
Jako počátek, kdy se poprvé SLAM objevil, se bere konference Robotics and Automation Conference, která se konala roku 1986 v San Francisco, California, USA, kdy byly pravděpodobnostní metody v začátcích, jak v robotice, tak v umělé inteligenci. Při této příležitosti se uskutečnila debata mezi odborníky v oblasti, z níž vzešlo, že se jedná o problém s mnoha koncepčními a výpočetními otázkami. \\
\indent K většímu posunu se dostalo o pár let později, kdy vyšla práce pojednávající o vztahu mezi landmarky (orientačními body) a zmenšení geometrické nepřesnosti. Důležitým prvkem zde bylo, že musí být velký stupeň korelace mezi odhady landmarků na mapě, přičemž korelace roste s postupným měřením. O tento posun se zasloužili pánové Smith, Cheesman a Duran-Whyte. \\
\indent Ve stejném období pracovali Nicholas Ayache a Olivier Faugeras pracovali na základech vizuální navigace a Crowley, Chatila a Laumond na navigaci pracující se sonarem používající Kalmanův filtr. Práce vznikly krátce po té Smithově a byly v základu dosti podobné. Tyto práce ukazovaly, že jak s robot pohybuje v prostoru sbírajíc relativní odhady landmarků, jsou tyto odhady ve vztahu k ostatním kvůli chybě v odhadu pozice robota. Důsledkem bylo zjištění, že pro získání správného řešení, je potřeba vytvářet stav složený z pozice robota a landmarků. Odhadující zařízení tak bude využívat velký stavový vektor s náročností výpočtu rostoucí v kvadrátu. Následovalo pak období, kdy se výzkumníci zaměřili na zlepšení problému mapování, kdy předpokládali minimalizaci, či úplné vymizení korelace mezi landmarky. Teoretická práce na kombinaci lokalizace a mapování se tak pozastavila a vývoj byl zaměřen odděleně pouze na jednu z částí.\\
\indent Později, když se poprvé formulovala kombinace lokalizace a mapování jako jeden problém, se zjistilo, že snaha minimalizovat korelaci landmarků byla chybná, naopak bylo poté v zájmu korelaci mezi landmarky co nejvíce zvýšit. Struktura SLAMu, a celkově první použití tohoto akronymu, byla prezentována v roce 1995 na International Symposium of Robotics Research (ISRR). Nezbytná teorie konvergence, s řadou počátečních výsledků, byla vytvořena Michaelem Csorbou, zatímco další týmy výzkumníků, například z MIT, Zaragozy nebo z ACFR ze Sydney, již započali práci na využití SLAMu (též nazývaného jako Concurrent mapping and localization = souběžné mapování a lokalizace (CML)) ve vnitřních, venkovních i podvodních systémech. V roce 1999 na ISRR se odehrálo první zasedání pojednávající přímo o SLAM a byla zda představena práce Sebastiana Thruna, ve které bylo dosaženo stupně konvergence mezi SLAMem založeném na Kalmanově filtru a pravděpodobnostním určením pozice a mapováním. Zájem o tento problém, tedy i pokroky v oblasti, od té doby exponenciálně rostly.\\
\subsection{Formulace a struktura SLAM} 
Jedná se o proces, při kterém robot vytváří mapu prostředí, v němž se pohybuje a na základě mapy určuje svoji pozici v prostoru. Pro určování trajektorie robota a rozložení landmarků není potřeba znalosti lokace robota. Odhad těchto parametrů probíhá součastně. 
\subsubsection{Legenda}
$k$ ... čas\\
$x_k$ ... stavovoý vektor popisující lokaci a orientaci robota\\ 
$u_k$ ... vektor aplikovaný v čase $k-1$ pro dostání robota v čase $k$ na pozici $x_k$\\
$m_i$ ... vektor popisující pozici i-tého landmarku, která je brána jako neměnící se v čase\\
$z_{ik}$ ... pozorování lokace i-tého landmarku z robota v čase $k$\\
$x_{0:k}$ ... historie umístění robota\\
$u_{0:k}$ ... historie vstupních dat pro pohyb robota\\
$m$ ... množina landmarků\\
$z_{0:k}$ ... množina pozorování landmarků
\subsubsection{Pravděpodobnostní SLAM}
V pravděpodobnostní formě řešení problému SLAM je zapotřebí výpočet rozložení pravděpodobnosti $P(x_k,m|Z_{0:k},U_{0:k},x_0)$ pro každý čas ${k}$. Toto rozložení popisuje srduženou posteriorní hustotu landmarků, stavu vozidla s ohledem na zaznamenané pozorování, řídící vstupy a zahrnuje čas ${k}$ s počátečním stavem vozidla. Obecně je žádoucí rekurzivní řešení, v tomto případě je pro výpočet využita Bayesův teorém, to má za následek, že je potřeba, aby přechodový stavový model a model pozorování navzájem popisovaly vliv vstupního řízení a pozorování.\\
\indent Model pozorování popisuje pravděpodobnost tvorby pozorování $z_k$, když je známá poloha vozidla a landmarků. Tato pravděpodobnost je popsána $P(z_k|x_k,m)$. Z toho lze předpokládat, že pokud je definována lokace vozidla a mapa, tak jsou pozorování, zhledem k mapě a momentálnímu stavu vozidla, podmíněně nezávislá. \\
Model pohybu vozidla může být popsán jako stavový přechodový model $P(x_k|x_{k-1},u_k)$. To znamená, že předpokládáme přechodový stav jako Markovův proces, při kterém následující stav $x_k$ je závislý pouze na předchozím stavu $x_{k-1}$ a aplikovaném řízení $u_k$ a není tak závislý ani na mapě, ani na pozorování.\\
\indent Nyní je SLAM implementovaný ve dvoustupňové rekurzivní formě korekce předpovědí. \\
Aktualizace času:
$$P(x_k,m|Z_{0:k-1},U_{0:k},x_0)=\int P(x_k|x_{k-1},u_k)\times P(x_{k-1},m|Z_{0:k-1},U_{0:k-1},x_0)dx_{k-1}$$ 
Aktualizace měření:
$$P(x_k,m|Z_{0:k},U_{0:k},x_0)=\frac{P(z_k|x_k,m)P(x_k,m|Z_{0:k-1},U_{0:k},x_0)}{P(z_k|Z_{0:k-1},U_{0:k})}$$
\subsubsection{Struktura pravděpodobnostního SLAMu}
Model pozorování $P(z_k|x_k,m)$ udává závislost pozování polohy vozidla a landmarků, což má za následek, že sdružená posteriorní hustota nemůže být klasicky rozdělena stylem \\$P(x_k,m|z_k)\neq P(x_k|z_k)P(m|z_k)$, toto rozdělení právě vede k chybným odhadům. Je také dáno, že spousta rozdílných odhadů od skutečné polohy landmarků je způsobena jediným zdrojem, čímž je chyba v odhadu, kde se robot v době pozorování nachází. Z toho vyplývá, že landmarky mezi sebou mají velkou korelaci a vzájemná odhadovaná pozice mezi 2 landmarky je tedy velice přesná i přes ne zcela přesný odhad jednoho z nich. V pravděpodobnostní formě to tedy znamená, že sdružená pravděpodobnostní hustota pro pár landmarků $P(m_i,m_j)$ je téměř zcela přesně dána, přičemž odhad bodu $P(m_i)$ nemusí být úplně přesný.\\
\indent Nejdůležitějším poznatkem bylo zjištění, že korelace mezi landmarky monotónně vzrůstá s počtem jejich pozorování, toto zjištění však bylo učiněno pouze pro lineární Gausův případ. Znamená to tedy, že odhad pozice všech landmarků $P(m)$ se bude stávat s narůstajícím počtem pozorování monotónně přesnější. Tento jev nastává díky, v podstatě, skoro nezávislému měření relativních pozic mezi landmarky, které jsou zcela nezávislé na natočení vozidla a úspěšné pozorovnání z tohoto bodu může nést další nezávislá měření relativních rozložení landmarků. \\
\indent Jak se robot pohybuje na další pozici $x_{k+1}$, opět zaznamenává lokaci landmarku $m_j$, díky které může určit pozici svoji a landmarku vůči své předchozí pozici $x_k$. Dále je pak aktualizována, díky vysoké korelaci mezi landmarky, pozice landmarku $m_i$ i v případě kdy již není vidět. V případě, že robot v bodě $x_{k+1}$ zpozoruje další dva landmarky, které budou vzhledem k $m_j$ dosud neznámé, jsou ihned spojeny nebo korelovány se zbytkem mapy. Pozdější jejich aktualizace tak budou ovlivňovat landmark $m_j$, tím pádem i $m_i$ a vytváří se tak síť navzájem propojených bodů, jejichž přesnost se zvyšuje s dalším pozorováním. Tento proces by se dal vizualizovat jako síť, propojující mezi sebou všechny landmarky, které jsou spojeny provázkem, jenž se dle vzájemné vzdálenosti a počtu měření zpevňuje. Opětovným projížděním prostorem se tedy tato síť stává robusnější a přesnější, omezením však stále zůstává kvalita mapy a měřícího senzoru.
\subsection{Řešení problému SLAM}
Při řešení je potřeba adekvátně obsáhnout jak složku modelace prostředí, tak i tvorbu pohybového modelu. Nejčastěji se setkáváme s reprezentací problému ve formě stavového modelu zatíženého šumem, což vede k použití rozšířeného Kalmanova filtru (v originále $extended$ $Kalman$ $filter\rightarrow EKF$). Jinou možností je rozčlenit pohybový model vozidla na vzorky s obecnějším negausovským rozdělením prtavděpodobnosti, což vede k použití Rao-Blackwellizedova partikulárního filtru. 
\subsubsection{EKF-SLAM}
Popis pohybu vozidla:
$$P(x_k|x_{k-1},u_k)\leftrightarrow x_k=f(x_{k-1},u_k)+w_k$$
$f(.)$ ... funkce modelující pohyb vozidla\\
$w_k$ ... chyby měření\\
$Q_k$ ... kovariance\\
\\
Model pozorování:
$$P(z_k|x_k,m)\leftrightarrow z_k=h(x_k,m)+v_k $$
$h(.)$ ... geometrické vlastnosti\\
$v_k$ ... chyby měření\\
$R_k$ ... kovariance\\
\\
Tyto dvě definice využijeme v EKF metodě k výpočtu průměru a kovariance sdruženě posteriorního rozložení\\
\indent průměr:
$$\begin{bmatrix}
\hat{x}_{k|k}\\\hat{m}_k
\end{bmatrix}=E
\begin{bmatrix}
x_k |Z_{0:k}\\m\indent
\end{bmatrix}$$ 
\indent kovariance:
$$P_{k|k}=\begin{bmatrix}
P_{xx} P_{xm}\\P^T_{xm} P_{mm}
\end{bmatrix}_{k|k}=E\begin{bmatrix}
\begin{pmatrix}
x_k-\hat{x}_k\\m-\hat{m}_k
\end{pmatrix} \begin{pmatrix}
x_k-\hat{x}_k\\m-\hat{m}_k
\end{pmatrix}^T |Z_{0:k}
\end{bmatrix}$$\\
Mezi hlavní problémy této metody se řadí konvergence, výpočetní složitost, sdružování dat a nelinerita. Prvním z nich, konvergence, se projevuje postupným přechodem determinantu mapové kovarianční matice $P_{mm,k}$ a podtříd dvojic landmarků k nule, jednotlivé odchylky landmarků pak konvergují dle původních nepřesností v odhadu pozice robota a jeho pozorování. Výpočetní složitost je brána v tuto chvíli jako kvadraticky rostoucí s počtem zaznamenaných landmarků, neboť při každém zaznamenaném pozorování se aktualizují již uložené landmarky, tento problém již ale prošel vývojem a existují již metody pracující v reálném čase s tisíci landmarky. Metoda EKF-SLAM je velmi náchylná na chybné spojení pozorování s landmarky. Jedná se zejména o problém s uzavřením smyčky, kdy dochází k opětovnému návratu na místo, ze kterého robot začínal. Nelinerita je posledním z významných problémů, kvůli nelinearitě můžeme dojít k větším nepřesnostem ve výsledku, neboť EKF-SLAM vzužívá lineárních modelů pro vyjádření nelineárního pohybu a modelu pozorování, konvergence a konzistence modelu je tedy jistá pouze v lineárním případě. 
\subsection{Rao-Blackweillizedův filtr}
Rao-Blackellizedův filt, jinak také FastSLAM, je zásadním posunem v návrhu rekuzivního pravděpodobnostního SLAMu. Předchozí pokusy pouze modifikovali EKF-SLAM, FastSLAM, na bázi rekurzivního Monte Carlo modelu, je prvním, který dokáže reprezentovat nelineání stavový model. Je výpočetně nemožné aplikovat na stavový prostor s vysokým počtem dimenzí partikulární filtry, je však možné redukovat velikost vzorků. \\
Sdružený stav SLAMu může být jako faktor komponentů vozidla a podmíněných komponentů mapy:
$$P(X_{0:k},m|Z_{0:k},U_{0:k},x_0)=P(m|X_{0:k},Z_{0:k})P(X_{0:k}|Z_{0:k},U_{0:k},x_0) $$ 
Rozdělení pravděpodobnosti zde není na jednotivých pozicích $x_k$, ale na celou trajektorii $X_{0:k}$ a tím se stávají jednotlivé landmarky na sobě nezávislými, mapa je tedy reprezentována jako soubor nezávislých gaussiánů, což znamená lineární složitost oproti kvadratické u čistého EKF-SLAMu. Hlavními ukazateli FastSLAMu je mapa, jež se vypočítává analyticky a vážené vzorky, kterými je reprezentována trajektorie pohybu. Rekurzivní odhad je proveden partikulárním filtrem pro stav pozice a EKF pro stav mapy. \\
\indent V této metodě se každý landmark zpracovává zvlášť, jeho pozice se aktualizuje stejným způsobem jako v EKF a landmarky, které nebyly zpozorovány, zůstávají na původní pozici a neaktualizují se. Vzájemnou neprovázaností landmarků však vzniká chyba v odhadu, která s časem roste, neboť jak se landamarky neovlivňují, nemohou tak aktualizací sítě landmarků napomoci k odhadu pozice robota a chyba odhadu jeho pozice tak stále roste.  \\
\indent Předpokládáme, pro obecnou formu FastSLAMu, že v čase $k-1$, je sdružený stav reprezentovaný jako $\{w^{(i)}_{k-1},X^{(i)}_{0:k-1},P(m|X^{(i)}_{0:k-1},Z_{0:k-1})\}^V_i$. V prvním kroce, pro každou částici vypočítáme návrh distribuce, jež je podmíněná svojí specifickou historií a z ní odebereme vzorek $x_k$, který je poté sdružen k historii částice $X^{(i)}_{0:k}$. Zahrneme zde i rozdíl mezi dvěmi možnými verzemi FastSLAM, jsou to FastSLAM 1.0 a FastSLAM 2.0. V tomto bodě se liší pouze formou navrhování distribuce. Krokem dva, dle funkce důležitosti, stanovíme váhy vzorků, v tomto případě je rozdíl verzí v důležitosti váhy pro další výpočty. Třetím krokem je případné převzorkování, které se provádí různě často dle implementace. Krokem posledním je provézt EKF update, na každou, již zpozorovanou částici, která je při aktuálním pozorováním zaznamenána. \\
\indent V dalším porovnání verzí FastSLAMu, verze 2.0 dává pro každou částici nejmenší možnou odchylku váhové důležitosti, vyplývá to z toho, že její návrh distribuce je lokálně optimální. Problém obou verzí je však neschopnost zapomínat minulost, dochází tak ke zrtátě statistické přesnosti, díky převzorkovávání, které přemazává historii měření, na níž je spolu s pozicí závislá tvorba mapy.   
\newpage
\section{Simultaneous localization and mapping: part II.}
\subsection{Výpočetní složitost}
Daný problém SLAM má neobvyklou strukturu, jelikož je odhad sdruženého stavu složený z pozice robota a landmarků, procesní model má tak vliv pouze na stav pozice robota a model pozorování na pár robot-landmark, a to vedlo k vytvoření spousty metod využívajících zmíněnou strukturu, pro redukci výpočetní složitosti. Základním rozdělením metod redukujících výpočetní složitost, je rozlišování optimálních, konzervativních a nekonzistentních metod. První typ, optimální metody, jsou založené na redukci daného výpočtu, výsledkem jsou pak odhady a kovariance, stejně tak, jako je tomu v případě plnohodnotné formy SLAM, rozebírané v předchozích kapitolách. U metod konzervativních dochází k odhadům s vyšší neurčitostí nebo kovariancí, většinou ale, i přes větší nepřesnost, jsou implementovány v reálném použití. Poslední možností jsou nekonzistentní metody. Jsou to algorithmy, které mají nižsí neurčitost nebo kovarianci, než algoritmy optimální. Pro řešení SLAM se však nepoužívají.\\
\indent Prvním přístupem pro redukci výpočetní složitosti, je využití struktury SLAM, pro omezení požadovaného výpočtu rovnicí aktualizace pozorování. Výpočet aktualizace času může být omezen metodami využívající rozšířený stav, výpočet stavu pozorování pak metodami oddělujícími rovnice dané aktualizace a obě tyto omezení vedou k redukci výpočtů, typicky jsou to optimální algoritmy. Další možnosí, jak snížit složitost procesu, je reformulace stavového prostoru do informační podoby, která umožnuje rozdělení výsledné matice s informacemi pro snížení výpočtů, což obvykle bývají algoritmy konzervativní. Obvykle je v díky nim znatelně redukována výpočetní složitost, stále však má dostatečně dobré odhadovací schopnosti. Dalším přístupem je submapping, který rozděluje mapu na regiony, kdy následné aktualizace můžou nastávat pouze v dané oblasti a s určitou periodou pak v i rámci ostatních oblastí. 
\subsubsection{Rozšířený stav}
Sdružený stavový vektor $x_k$, v čase $k$, se skládá z pozice robota a jím zaznamenaných landmarků, kdy model robota ovlivňuje pouze stav pozice, a to vlivem vstupního řízení, stav mapy se tím tedy nemění. 
$$x_k=\begin{bmatrix}
f_v(x_{vk-1},u_k)\\m
\end{bmatrix} $$
Pokud při volbě typu SLAM jako EKF, má výpočet předpovědi kovariance kubicky rostoucí složitost s počtem landmarků
$$P_{k|k-1}=\bigtriangledown f_xP_{k-1|k-1}\bigtriangledown f^T_x+\bigtriangledown f_uU_k\bigtriangledown f^T_u,$$
to se však dá předělat na formu s pouze lineární složitostí
$$P_{k|k-1}=\begin{bmatrix}
\bigtriangledown f_{vx}P_{vv}\bigtriangledown f^T_{vx}+\bigtriangledown f_{vu}U_k\bigtriangledown f^T_{vu}&&\bigtriangledown f_{vx}P_{vm}\\
P^T_{vm}\bigtriangledown f^T_{vx}&&P_{mm}
\end{bmatrix} $$
Přidání nového landmarku má podobný tvar, kdy je nový landmark inicializován jako funkce pozice robota a pozorování a rozšířený stav pak získáme z malého množství existujících stavů 
$$x^+_k=\begin{bmatrix}
x_{vk}\\m\\g(x_{vk},z_k)
\end{bmatrix},$$
kde $g(x_{vk},z_k)=m_{new}\rightarrow$ přidání nového landmarku \\
Rozšíření stavu můžeme aplikovat vždy, když je nový stav funkcí podmnožiny již existujících stavů
$$\begin{bmatrix}
x_1\\x_2\\f(x_2,q)
\end{bmatrix}, $$
$$
\begin{bmatrix}
P_{11}&&P_{13}&&P_{13}\bigtriangledown f^T_{x_2}\\
P^T_{12}&&P_{23}&&P_{23}\bigtriangledown f^T_{x_2}\\
\bigtriangledown f_{x_2}P^T_{11}&&\bigtriangledown f_{x_2}P_{32}&&\bigtriangledown f_{x_2}P_{32}\bigtriangledown f^T_{x_2}+\bigtriangledown f_qQ\bigtriangledown f^T_q
\end{bmatrix}
$$
\subsubsection{Rozdělené aktualizace}
Jedná se o metody vytvářející optimální odhady. Při implementaci základní podoby aktualizace pozorování, se při každém novém měření aktualizuje jak stav vozidla, tak i mapy, což vede ke kvadratickému nárůstu složitosti s množstvím landmarků. Při této metodě si však rozdělíme mapu na menší oblasti, které se aktualizují při průjezdu robota danou částí, zatímco aktualizace celé mapy probíhá s výrazně nižší frekvencí.\\
\indent Rozlišujeme dva způsoby možné implementace. První z nich pracuje na zmenšené oblasti, ale stále si drží globální referenční souřadnice, jedná se například o algoritmus CEKF (compressed EKF). Druhou možností je tvorba menších map s vlastním souřadnicovým rámcem neopouštějícím danou submapu, jedná se o algorithmy CLSF (constrained local submap filter $\rightarrow$ omezený lokální submapový filtr). Pokračovat budeme v rozboru druhé možnosti, neboť je jednodušší a při provádění operací s velkou frekvencí opakování je méně ovlivněna linearizačními chybami, je stabilnější a zabraňuje příliš velkému nárůstu globální kovariance. \\
\indent Logaritmus submapy se skládá z dvou nezávislých odhadů, které si stále udržuje. Jde o vektory $x_G$ a $x_R$, kdy $x_G$ je mapa složená z globálně referencovaných landmarků a globálně referencované pozice dané submapy a $x_R$ je lokální submapa s lokálně referencovanou pozicí robota lokálně referencovanými landmarky. Při získání pozorování se aktualizují pouze landmarky náležící aktuální submapě, ve které se robot nachází. Celkový globální odhad pak získáváme periodicky, zaevidováním submapy do mapy celé a použitím aktualizace omezení na společné vlatnosti obou map.
\subsubsection{Rozčlenění}
Do této doby jsme brali stavový odhad $\hat{x}_k$ a matici kovariance $P_k$, jako produkty klasicky implementovaného EKF-SLAM, ty pak spolu popisovaly první dva centrální momenty hustoty Gaussovy pravděpodobnosti skutečného stavu $x_k$. Jinou možností je vyjádřit stavový odhad a matici kovariance v informační formě pomocí matice informací $Y_k=P^{-1}_k$ a vektoru informací $\hat{y}_k=Y_k\hat{x}_k$. Tato změna je výhodná pro mapy s větším měřítkem, kdy spousta nediagonálních prvků bude velmi blízkých nule, což vede k možnosti nastavení těchto prků na hodnotu nula, přičemž však může vznikat malá ztráta optimality při vzniku map. \\
\indent Důležitým poznatkem je fakt, že rozšíření stavu je rozčleňovací operace vedoucí k přesnému rozčlenění informační formy SLAMu a má tak ekvivalentní informační formu, kde pro zjednodušení považujeme šum za nulový přídavek $f(x_2,q)=f(x_2)+q$. Podmnožinu stavů $x_1$ je pak předpokládána jako obsahující většinu stavů mapy a po rozčlenění, dosahuje pouze konstantní složitosti v čase. Můžeme tedy získávat přesné řešení rozšiřováním stavu novým odhadem pozice robota v každém kroce a zachovat všechny předchozí pozice. Díky tomu jsou nenulové nediagonální prvky pouze ty, které jsou spojené napřímo s měřenými daty.  \\
\indent Dále sem musíme zahrnout marginalizaci, jež je nezbytná pro odstranění předchozích stavů pozice. Máme možnost marginalizovat všechny předchozí stavy, což vede na zhuštěnou matici informací, to je stav, kterého dosáhnout nechceme. Správnou vlbou ukotvení pozice můžeme marginalizovat velkou čast pozic, aniž bychom vyvolaly nadměrnou hustotu matice informací. Pro reálné využití v praxi se zde stále vyskytuje problém s obnovováním průměru a kovariance v každém kroce, což může být velmi výpočetně náročné, dá se však poměrně efektivně získat konjugačními gradientními metodami. 
\subsubsection{Globální submapy}
U globálních metod rotzlišujeme základní dva typy, globálně a lokálně referencované, přičemž oba mají společné, že submapa stanovuje místí souřadnicový rámec a landmarky z jejího okolí jsou odhadovány s ohledem na daný místní rámec. \\
\indent Lokální metoda získává odhady pomocí optimálního algoritmu SLAM, používajícího pouze lokální landmarky. Tato metoda, i přes výpočetní efektivitu, je při tvorbě struktury submap neoptimální. Globální metoda je mnohem zajímavější. Dokáže z kvadraticky rostoucí složitosti, udělat lineární, či dokonce v čase konstantní složitost. Je to možné díky údržbě a konzervativním odhadům celkové mapy. Metoda stojí na odhadování globální pozice submapy v rámci společného základního rámu, nevede však ke zmírnění problémů s linearizací, způsobenou velkými nepřesnosti v pozicích.
\subsubsection{Submapy vztažné}
Základním rozdílem od metody globálních submap, je absence společného základního rámu. V této metodě se submapy zaznamenávají dle sousedství s ostatními a celkovou mapu pak můžeme získat souhrnem vektoru cestou po vytvořené síti submap. Submapy jsou, díky vyhýbání se globálním spojením, velmi zajímavé z hlediska výpočetní složitosti a problémům týkajících se nelinearity. Velkým kladem je například tvorba lokálně optimální mapy s výpočetní složitostí nezávislou na celkové velikosti mapy a dále, díky úpravám pouze na lokální úrovni, je velmi stabilní.
\subsection{Sdružování dat}
Jedná se velmi důležitý problém, neboť i když během procesu tvorby mapy, dojde k jedné chybné asociaci dat, může to vést k destabilizaci odhadu mapy, často dokonce k pádu celého algoritmu
\subsubsection{Validace várky dat}
Z prvu se k problému přistupovalo způsobem, kdy každé jednotlivé zachycení landmarku, se porovnávalo se všemi odhady nacházejícími se v blízkém okolí, tento individuální přístup je neproveditelný, pokud je pozice robota velmi nejistá, což znamená selhání ve všech, obzvlášť ve strukturovaných a málo osídlených prostředích. Později se již podařilo násobné asociace zvažovat simultánně. 
\subsubsection{Popis vzhledu}
Jedním ze způsobů snímání okolí je vidění, kdy zaznamenáváme tvar, barvu, strukturu, a tím dokážeme rozlišovat různé balíčky data, což v SLAMu využíváme pro přepověd dané asociace. Nejčastěji tak pro problém s uzavřením smyčky, ke kterému se dostalo z původního využití, a to rozpoznání míst v topologických mapách a indexování databáze obrázků. Pokrok této metody přišel s prací Newmana a spol., kdy se začala počítat metrika podobnosti přes sekvenci obrazů, místo původního jednoho a pro použití v běžném režimu je použita metoda valstních čísel. 
\subsubsection{Multihypoziční sdružování dat}
Jedná se o metdu nezbytně potřebnou pro robustní sběr cílů v zaplněném prostředí, kdy tvorbou oddělených odhadů trasy jízdy pro každou asociační hypotézu, reší problém nejednaznočni hypotéz. Tato funkce je však silně limitována dotupným výpočetním výkonem. Dale je metoda vyžívána, zvláště ve velkých prostředích, v implementaci robustního SLAMu, kdy je při uzavírání smyčky vytvořena hypotéza pro smyčku uzavřenou, tak i pro stále neuzavřenou, a tím se bere v potaz, že je prostředí pouze podobné. 
\subsection{Reprezentace prostředí}
Původně se modeloval svět pomocí SLAM pouze jako soubor landmarků, majících svůj určitý tvar, později však, zejména ve venkovním, podvodním a podzemním použití, se ukázala tato metoda jako nevyhovující.
\subsubsection{Částečná pozorovatelnost a zpožděné mapování}
Základními dvěma typy pro pozorování je vidění, pomocí kamery, nebo nějaký typ dálkového senzoru. Kamera, pokud je na robotovi samostatně, zazamenává informace neobsahující přesnou vzdálenost objektu. Tento problém měření pomocí senzoru nemá, neboť meření vzdálenosti bývá velmi přesné, za to se pro tento typ vyskytuje problém se šířkou vysílaného paprsku a postraními žlábky. S tímto senzorem nedokážeme jedním pozorováním ani vytvořit přesně položený landmark, jedním měřením totiž získáváme negaussovské rozložení pro pozici landmarku a potřebujeme tak větší počet jeho zpozorování pro vytvoření odhadu.\\
\indent Generalizované rozložení umožňují nezpožděné sledování landmarků, zpožděním inicializace však můžeme získat hned gaussovský odhad pozice landmarku, je při tom ale potřeba zaznamenávat pozici robota v každém okamžiku měření, což provedeme rozšířením stavu o vektor pozic. Nejedná se pouze o částečnou pozorovatelnost, ale shromažďováním informací a zpožděným rozhodováním, zvyšuje robustnost procesu. 
\subsubsection{3D SLAM}
Jedná se v podstatě pouze o rozšíření 2D SLAMu, které má však výrazně větší výpočetní složitost a komplikovanější modelování, způsobené složitějším snímáním. Rozlišujeme základní tři možnosti. První je klasický 2D SLAM s přidanou schopností vytvářet třetí dimenzi, což se užitečné, pokud se robot pohybuje po rovině. Druhý typ vytváří 3D obraz extrakcí diskrétních landmarků a sdruženého odhadu pozice vozidla a mapy. Využití je vhodné v robotech s jedním senzorem, který umožnuje pohyb se šesti stupni volnosti. Poslední možnost se od těch předchozích dost odlišuje. Sdružený stav se totiž skládá z předchozích pozic robota, na každé pozici se udělá 3D sken prostředí a odhad pozice se vyrovná korelaí skenů. 
\subsubsection{Trajektorií orientovaný SLAM}
Základní formulací problému, je odhadovaný stav jako pozice robota a zaznamenané landmarky, jinou, novější, možností je odhadovat místo toho trajektorii vozidla. Mapa tak není součástí stavu, každá pozice robota má ale přidružený sken svého okolí, které se pak srovnají a vytvoří globální mapu. Z toho je možno vidět jeden z problémů, což je neustálý růst stavového prostoru, který se nijak nepromazává. Využití se nachází například při tvorbě topologickým map.
\subsubsection{Vložené pomocné informace}
Jak je vidět u trajetoriemi orientovaného SLAMu, můžeme ke stavu připojit data, v tomto případě to dále rozšířit například o teplotu, charakteristiku povrchu a mnoho dalších věcí, které pak napomáhají při tvorbě mapy. Tvorba takovéto struktury je ale poměrně komplikovaná. Pohybem robota prostředím, ukládají se pomocná data do datové struktury tak, že každá buňka v této struktuře je přiřazena daným landmarkům v mapě, při aktualizaci jejich pozice se tak s nimi přemisťují i jim přidružené informace. 
\subsubsection{Dynamická prostředí}
Ve světě se setkáváme převážně z dynamicky se vyvíjejícím prostředím, kdy nám do pozorování můžou zasahovat lidé, zvířata, nebo třeba nábytek jako židle, nebo zaparkovaná auta. Musí se tedy nějak určit, co s takovými objekty bude a jak je za pohyblivé určit. Máme možnost tyto objekty do mapy vůbec nepřidávat, nebo je mít označené a pohyblivé, nesmí se ale stát, že zaznamenáme pohyblivý objekt a uložíme ho jako statický. \\
\indent Klasická implementace SLAM umí odstranit landmark, i poměrně velké mnořství landmarků, bez většího vlivu na hodnotu konvergence. To je také využíváno pro úpravu mapy, kdy se odstraňují přebytečné landmarky, které se již na svém místě nenacházejí. 

\section{Improved techniques for grid mapping with Rao-Blackwellized particle filters}
\subsection{Introduction}
- tvorba mapy - základní funkce mobilního robota (problém mapování = SLAM)\\
- pro odhad pozice dobrá mapa, pro získání mapy dobrý odhad pozice\\
- $\rightarrow$ RBPF - effective solution\\
- redukce dat, převzorkování může elimiovat správné částice $\rightarrow$ problém vyčerpání částice\\
- 2 přístupy zlepšení RBPF \\
- - návrhové rozložení zahrnuje přesnost senzoru $\rightarrow$ přesné vykreslení částic\\
- - vzorkovací technika udžující rozumné množství částic, umožňuje algoritmu určovat přesnou mapu a snižuje riziko vyčerpání částic\\
- návrhové rozložení - vyhodnocováním pravěpodobné pozice (kombinace informací z laseru a odometrie), poslední pozorování použito pro tvorbu nových částic $\rightarrow$ odhad stavu na základě více informací než pouze na odometrii \\
- - přesnější mapa - zvážení vlivu pozorování na pozici, poté až změny v mapě $\rightarrow$ redukce chyby odhadu\\
- - adaptivní vzorkování - udělání zvorku pouze když ho potřebujeme, udržení rozumné různosti částic $\rightarrow$ snížení rizika vyčerpání částice\\
\subsection{Mapping with RBPFs}
- základní úkol - odhad sdružené postreriorní pravděpodobnosti\\
- - cíl - mapa, trajektorie\\
- - odhad na základě - pozorování, odometrie\\
- odhad trajektorie, poté až mapy\\
- mapy tvořeny pozorováním, trajektorie odpovídajícími částicemi (každá částice reprezentuje část trajektorie)\\
- částicové filtry - SIR (sampling importance resampling) filter, pro mapovánní postupně zpracovává data ze sezoru, pak odometrii, updatuje sadu vzorků reprezentujících posterior o mapě a trajektorii\\
- - vzorkování - nové částice jsou získávány z předchozí generace odběrem vzorků z návrhového rozložení (často pravděpodobnostní odometrie) 
- - vážená důležitost - individuální váhu důležitosti má každá částice podle zásady důležitostního vzorkování. Váhu zavádíme kvůli tomu, že cílové rozložení neni rovno tomu navrhovanému.\\
- - převzorkovávání - částice jsou přepisovány úměrně jejich vážené důležitosti, nezbytný krok pro potřebnou konečnost částic k aproximaci kontinuálního rozložení, dále převzorkování umožňuje použít částicový filtr v situaci neschodujícího se cílového rozložení od toho navrhovaného (po převzorkování všechny částice mají stejnou váhu)\\
- - odhad mapy - pro každou částici je vypočítána mapa na základě trajektorie vzorku a historie pozorování\\
- pro implementaci potřeba vyhodnocovat váhy trajektorií při každém pozorování, po delší době neefektivní\\
- omezením navrhovaného rozložení získáme rekurzivní formulaci pro výpočet vážené důležitosti\\
\subsection{RBPF with improved proposals and adaptive resampling}
- zlepšení návrhového rozložení - pro zisk nové generace částic je potřeba vykreslení vzorků z návrhového rozložení (čím lepší návrh tím lepší výsledek, kdyby přímo jako cílové rozložení $\rightarrow$ stejné vážené důležitosti částic, není potřeba převzorkování)\\
- - typický částicový filtr - návrhové rozložení = odometrický pohybový model (+ jednoduchý výpočet pro většinu robotů, vážená důležitost je počítána podle modelu pozorování) - toto rozložení není optimální, zvlášť pokud je senzor výrazně přesnější než odhad pozice (laser)\\ 
- - vyhlazování pravděpodobné funkce - zabránění, částicím v blízkosti významné oblasti, přílišného poklesu vážených důležitostí (- vyřazuje užitečnou informaci ze senzoru $\rightarrow$ nepřesnější mapa)\\
- - překonání problému - zvažovat poslední pozorování při generování nových vzorků, integrováním posledního pozorování do návrhu se dá zaměřit na vzorkování ve významné oblasti pravděpodobnosti pozorování\\
\\
- efektivní výpočet zlepšeného návrhu - nepředvídatelná funkce pravděpodobnosti pozorování $\rightarrow$ nedostupná uzavřená aproximace informačního návrhu\\
- - teorie - získání aproximace upraveným částicovým filtrem - návrh pro každou částici je vypočten odhadem vzorků z optimálního návrhu\\
- - SLAM - navzorkování potenciální pozice z pohybového modelu $\rightarrow$ zvážení vzorků podle pravděpodobnosti pozorování (zisk aproximace optimálního návrhu - pokud již je, stále je potřeba vysoký počet vzorků pozice z pohybového modelu)\\
- - většinou cílové rozložení omezený počet maxim (většinou jedno) $\rightarrow$ vzorkování pozic kolem maxima (snížení složitosti) - zvažujeme pravděpodobnost pozorování i pohybový model v době použití senzoru - lokální aproximace posteriorní pravděpodobnosti kolem maxima pravděpodobnostní funkce\\
- - - na těchto datech počítáme Gaussovskou aproximaci - efektivní vykreslování nových vzorků\\
- - - porovnání scanů pro zisk významé oblasti z funkce pravděpodobnosti pozorování $\rightarrow$ navzorkujeme tuto oblast, vyhodnotíme vzorky podle cílového rozložení\\
- - - pro každou částici $i$ jsou Gaussiánské parametry ($\mu_t^i$, $\Sigma_t^i$) určené individuálně pro daný počet vzorků na intervalu kdy je aktivní senzor + bereme v potaz odometrii při výpočtu střední hodnoty $\mu^i$ a výchylky $\Sigma^i$ $\rightarrow$ zisk uzavřené aproximace optimálního návrhu $\rightarrow$ efektivní zisk nové generace částic - pro výpočet váhové funkce používáme stejný normalizační faktor jako u Gaussiánské aproximace\\
\\
- zlepšený návrh - získání parametrů Gaussiánského návrhu zvlášť pro každou částici (uvažuje odometrii a pozorování, mezitím efektivně vzorkuje), snížená neurčitost výsledných hustot\\
- - porovnávač pozorování určuje režim významné oblasti funkce pravděpodobnosti pozorování - vzorkování z důležitých oblastí - většina porovnávačů maximalizuje pravděpodobnost pozorování, dává mapu a nástřel pozice robota\\
- - vícerežimová pravděpodobnostní funkce (uzavření smyčky) - porovnávač vrátí pro každou částici maximum nejblíže původnímu odhadu - může způsobit ztrátudalších maxim v pravděpodobnostní funkci, protože je hlášen jen 1 režim\\
- - přílišná důvěřivost filtru (teoreticky) - přeplněný prostor + odometrie silně zatížena šumem\\
- - - řešení - sledovat více režimů porovnávače měření + opakovat vzorkování v každém uzlu\\
\\
- adaptivní převzorkování\\
- - převzorkování - důležitý aspekt výkonu částicového filtru, nahrazování vzorků s nízkou váhou důležitosti těmi s vysokou\\
- - - nezbytné - konečný počet částic je použit k aproximaci cílového rozložení (- může odstanit dobré vzorky z filtru $\rightarrow$ ochuzení částic) $\rightarrow$ důležité převzorkování ve správný čas, mít dobré rozhodovacví kritérium\\
- - Liu - efektivní velikost vzorkování pro odhad, jak aktuální sada částic reprezentuje cílový posterior\\
$$N_{eff}=\frac{1}{\varSigma_{i-1}^N(w^{(i)})^2}$$ 
- - - $w^{(i)}$ ... normalizovaná váha částice $i$ \\
- - - $N_{eff}$ - vzorky z cílového rozložení $\rightarrow$ stejné váhové důležitosti (zhoršující se aproximace cílového rozložení $\rightarrow$ větší ochylka váhových důležitostí)\\
- - - - Doucet - převzorkování, když $N_{eff}$ < N/2 (N ... počet částic) - výrazná redukce možnosti nahrazení dobrých částic (redukován počet převzorkování, převzorkování když je potřeba)\\
 - - algoritmus\\
 - - - nástřel pozice reprezentovaný danou částicí, získáno z předchozí pozice této částice a odometrického měření od poslední aktualizace filtru\\
 - - - na základě mapy provedeno porovnání pozorování z místa úvodního nástřelu pozice, vyhledává se pouze v okolí tohoto bodu, v případě selhání - pozice a váhy počítány podle pohybového modelu (další 2 kroky přeskočeny)\\
 - - - vybrána sada vzorků kolem dané pozice, vypočítání průměru a kovarianční matice návrhu bodovým hodnocením cílového rozložení v pozici vzorku, také počítán váhový faktor\\
 - - -  nová pozice částice je zakreslena podle z Gaussovské aproximace podle zlepšeného návrhového rozložení\\
 - - - aktualizace váhových důležitostí\\
 - - - mapa částice je aktualizována podle její zakreslené pozice a pozorování\\
\subsection{Implementation issues}
 - porovnávač "vasco" - hledání pozice porovnáním scanu vůči mapě se zanesením i úvodního odhadu (sestupné gradientní hledání v pravděpodobnostní funkci aktuálního pozorování vůči grid mapě - hledání maxima funkce)\\
 - - může být použit každý porovnávač umožňující najít nejlepší zarovnání mezi mapou a aktuálním pozorování s ohledem na úvodní odhad\\
 - Bayesovo pravidlo - hledání pozice\\
 - model koncového bodu - výpočet pravděpodobnosti pozorování\\
 - - paprsky nezávislé - pravděpodobnost paprsku počítána na základě vzdálenosti konce paprsku od nejbližší překážky od tohoto bodu - pro rychlý výpočet využití konvolované lokální grid mapy\\
 - návrh - výpočet 2 komponent - jejich vyhodnocování pro každý vzorek\\
 - - 1. - výpočet podle modelu koncového bodu\\
 - - 2. - Gaussovská aproximace odometrie pohybového modelu - aproximace získána přes rozšíření Taylorovy řady\\
 - 0.5 m mezi 2 použití filtru - aproximace funguje dobře, žádný rozdíl mezi modelem na EKF a přesnějším  modelem rychlosti pohybu na bázi vzorků\\

\subsection{Complexity}
Výpočet návrhového rozložení ... O(N)\\
Aktualizace grid mapy ... O(N)\\
Výpočet vah ... O(N)\\
Testováí potřeby převzorkování ... O(N)\\
Převzorkování ... O(N*M)\\
 - naučení se grid map s RBPF\\
 - - sada vzorků - posteriorní pravděpodobnost o mapách a pozicích\\
 - - počet vrozků - centrální množství\\
 - - návrhové rozložení - vzorkování u nejpravděpodobnější pozice danou porovnávačem pozorování\\
 - - vzorkování pro každou částici, žádná závislost mezi nimi při výpočtu návrhu\\
 - - poslední pozorování - výpočet Gaussiánských parametrů\\
 - - - kryje část mapy $\rightarrow$ komplexita závisí jen na počtu vzorků\\
 - - - - to samé platí pro aktualizaci jednotlivých map spojených s každou částicí\\
 - - kopírování informace náležící částici při převzorkování - nejhůže N-1 částic nahrazeno jednou\\
 - - každá částice si nese svou grid mapu $\rightarrow$ duplikování částice - kopírování její mapy $\rightarrow$ převzorkování má složitost O(N*M) - M ... velikost grid mapy\\
 - - - používání adaptivního převzorkování\\

\subsection{Experiments}
 - ActivMedia Pioneer2 AT, Pioneer 2 DX-8, iRobot B21r\\
 - venkovní i vnitřní prostory\\
 - až rozlišení do 1 cm bez problémů\\
 - 250mx250m - max 80 částic\\
 - výsledky mapování \\
 - - Intel Research Lab - vnitřní prostor, 28mx28m, 15 částic, Pioneer II, SICK senzor, velmi přesné\\
 - - Freiburg Campus - venku, 250mx250m, 30 částic, keře,stromy, pohyblivé předměty (auta, lidi) i tak přesná mapa\\
 - - MIT Killian Court - vnořené smyšky (potenciální vyčerpání částic), dlouhé chodby, důležitá vlastnost adaptivního převzorkování, 60 částic (občas dvojité stěny), 80 částic (vysoká kvalita)\\
 - vyčíslitelné výsledky - porovnání počtu částic s přístupem dirka Hahnela, počet částic které potřebuje RBPF pro vytvoření topologicky přesné mapy alespoň v 60\% použití - Intel (8 vs. 40), Freiburg (20 vs. 400), MIT (60 vs. 400) - mnohem méně částic + lepší mapa (lepší vzorkování) \\
 - účinky lepších návrhů a adaptivního vzorkování\\
 - - zlepšené návrhové rozložení - generování vzorků s vyskovou pravděpodobností, \\
 - - adaptivní vzorkování - $N_{eff}$\\
 - - návrhy bez celé vstupní historie snižují v čase $N_{eff}$ (čím horší návrh, tím rychleji) - obnovení na max hodnotu po převzorkování\\
 - - v této implementaci\\
 - - - zlepšení poklesu $N_{eff}$ - klesá pomaleji i v neznámém prostředí (návrhové rozložeí není tak vypíchnuté)\\
 - - - částice při průchodu známým prostředím zůstávají ve vlastní mapě (díky zlepšenému návrhovému rozložení a kvůli váhám)\\
 - - - uzavírání smyčky - některé částice jsou správně seřazeny (velká váha) a některé chybně (nízká váha) $\rightarrow$ zvětšení výchylek vah $\rightarrow$ snižování $N_{eff}$ - většinou vynucení převzorkování\\
 - - - výrazné snížení pravděpodobnosti vyčerpání částice - uchovávání vzorků v sadě částic\\
 - vliv odometrie na návrh - většinou je návrh jen z laseru dobrý pro odhad pohybu částic, někdy potřeba odometrie pro návrhové rozložení - např. slabé vlastnosti dat z laseru (pustá místa, koridory)\\
 - situace, kdy selhá porovnávač pozorování - nemusí na základě dat dojít k odhadu pozice\\
 - - vzorkujeme rovnou z odometrie pro zisk nových částic\\
 - - vevnitř se tento problém nestává, venku v otevřených prostorách, když laser nikde ve svém rozsahu nic nezaznamenává\\
 - analýza běhu - PC 2.8 GHz procesor, scan při pohybu o 0.5m nebo otočení o $25^o$\\
 - - výpočet návrhového rozložení, vah, aktualizace mapy (1910 ms), testování potřeby převzorkování (41 ms), převzorkování (244 ms)\\



\end{document}