\documentclass[11pt]{article}
\usepackage{graphicx}
\usepackage[czech]{babel}
\usepackage[utf8]{inputenc}
\usepackage{titling}
\usepackage{pdfpages}
\usepackage[nopar]{lipsum}
\usepackage{mathtools}
\usepackage{multirow}
\usepackage{caption}
\usepackage{float}
\usepackage{enumitem}
\usepackage{listings}
\usepackage{amsmath}
\usepackage{amssymb}
\setlength{\parindent}{1cm}
\setlength{\hoffset}{-1.1cm} 
\setlength{\voffset}{-2cm}
\setlength{\textheight}{23.0cm} 
\setlength{\textwidth}{15cm}

\begin{document}
\title{Bakalářská práce}
\author{Lukáš Kuhajda}
\date{Akademický rok 2018/2019}
\begin{titlepage}
	\begin{center}
		\includegraphics[scale=0.5]{logo_zcu}\\
		\vspace{5cm}
		\begin{Large}
			\textbf{\thetitle}\\
		\end{Large}
		
		\vspace{3cm}
		\theauthor\\
		\vspace{5cm}
		\thedate
	\end{center}
\end{titlepage}
\newpage	
	
\tableofcontents
\newpage





\section{Úvod}
Ve své bakalářské práci se budu věnovat systémům, které pomocí měření z LIDARu (Light Detection And Ranging) utváří mapu prostředí, v němž se pohybují. V první části práce jsem se zaměřil obecně na problém simultánní lokalizace a mapování (SLAM). Mobilní robot je umístěn do neznámého prostředí a jeho úkolem je určovat svoji pozici a utvářet mapu. Zabývám se zde vnitřními principy a různými přístupy k řešení problému. Daná tématika momentálně vstupuje do podvědomí i širší veřejnosti, neboť pomalu dochází k přechodu na autonomní vozidla, která fungují na podobných principech. Tyto automobily však využívají i mnoho dalších senzorů, jako jsou například kamery. Já se zaměřuji na systémy využívající čistě jen 2D LIDAR, tedy o sezor měřící vzdálenosti pouze v jedné výškové úrovni.\\
\indent V další části své práce jsem se podrobněji věnoval třem nejrozšířenějším 2D SLAM systémům, přesněji se jedná o GMapping, HectorSLAM a Google Cartographer. Popsal jsem zde jejich základní rysy, kterými se liší od ostatních a jimiž je dostatečně popsána jejich funkčnost. \\
\indent měření, výsledky, porovnání, změna v gmappingu, proč jsem si vybral ...

\newpage

\section{Simultánní lokalizace a mapování}

\subsection{Historie}
Jako počátek řešení problému se považuje konference Robotics and Automation Conference konaná v roce 1986. Pravděpodobnostní metody byly tehdy ještě velmi nerozvinuté, jak v robotice, tak i v umělé inteligenci. Došlo tedy pouze k debatě na dané téma.\\
\indent K většímu posunu kupředu se dostalo o pár let později, kdy vyšla práce pojednávající o vztahu mezi orientačními body (landmarky) a zmenšení geometrické nepřesnosti. Důležitým prvkem zde bylo zjištění, že mezi odhady landmarků na mapě je velká korelace, které je rostoucí s dalšími pozorováními.\\
\indent Ve stejném období vznikaly základy vizuální navigace a navigace pracující se sonarem s použitím Kalmanova filtru. Práce byly v základu dosti podobné. Ukazovaly, že odhady landmarků získané pohybem robota prostředím, jsou v korelaci s ostatními kvůli chybě v odhadu pozice robota. Je tak třeba mít stav složený z pozice robota a landmarků, tím však vznikal velký stavový vektor s náročností rostoucí v kvadrátu. V daném období byla tendence korelaci landmarků snižovat.\\
\indent Později došlo k sjednocení problémů lokalizace a mapování a závěru, že snaha minimalizovat korelaci mezi landmarky byla chybná, naopak bylo v zájmu korelaci co nejvíce zvýšit. Struktura SLAMu, a celkově první použití tohoto akronymu, byla prezentována v roce 1995 na International Symposium of Robotics Research (ISRR). Poté se v roce 1999 na ISRR odehrálo první zasedání pojednávající přímo o SLAM a došlo k představení práce dosahující dostatečné konvergence mezi SLAMem na bázi Kalmanova filtru a pravděpodobnostními metodami pro lokalizaci a mapování.


\subsection{Formulace a struktura} 
Jedná se o proces, při kterém robot vytváří mapu prostředí, v němž se pohybuje a na základě mapy určuje svoji pozici v prostoru. Pro určování trajektorie robota a rozložení landmarků není třeba znalosti jeho lokace, neboť odhad těchto parametrů probíhá součastně. 

\subsubsection{Pravděpodobnostní SLAM}
V této formě je cílem získat odhad hustoty pravěpodobnosti
\begin{equation}
	P(\textbf{x}_k,\textbf{m}|\textbf{Z}_{0:k},\textbf{U}_{0:k},\textbf{x}_0)
\end{equation}
pro každý časový okamžik ${k}$, kde $\textbf{x}_k$ je stavový vektor popisující pozici a orientaci robota, $\textbf{m}$ je množina landmarů, tedy mapa, $\textbf{Z}_{0:k}$ jsou všechna pozorování landmarků, $\textbf{U}_{0:k}$ je historie vstupů a $\textbf{x}_0$ je počáteční stav. Jedná se tedy o srduženou posteriorní hustotu mapy, stavu vozidla s ohledem na zaznamenané pozorování, řídící vstupy a zahrnuje čas ${k}$ s počátečním stavem robota. Pro výpočet je využit Bayesův teorém, je tedy potřeba, aby pohybový model a model pozorování popisovaly vliv vstupního řízení a pozorování.\\
\indent Model pozorování popisuje pravděpodobnost zisku pozorování $\textbf{z}_k$, pokud známe polohu vozidla a landmarků.
\begin{equation}
	P(\textbf{z}_k|\textbf{x}_k,\textbf{m})
\end{equation} 
\indent Model pohybu vozidla může být popsán jako stavový přechodový model. \\Předpokládáme přechodový stav jako Markovův proces, při kterém následující stav $\textbf{x}_k$ je závislý pouze na předchozím stavu $\textbf{x}_{k-1}$ a aplikovaném řízení $\textbf{u}_k$ a není tak závislý ani na mapě, ani na pozorování.
\begin{equation}
	P(\textbf{x}_k|\textbf{x}_{k-1},\textbf{u}_k)
\end{equation}
Máme tak implementaci ve formě dvoustupňového rekurzivního algoritmu.\\
\\
Aktualizace času (predikce):
\begin{equation}
	P(\textbf{x}_k,\textbf{m}|\textbf{Z}_{0:k-1},\textbf{U}_{0:k},\textbf{x}_0)=\int P(\textbf{x}_k|\textbf{x}_{k-1},\textbf{u}_k)\times P(\textbf{x}_{k-1},\textbf{m}|\textbf{Z}_{0:k-1},\textbf{U}_{0:k-1},\textbf{x}_0)d\textbf{x}_{k-1}
\end{equation}
\\
Aktualizace měření (korekce):
\begin{equation}
	P(\textbf{x}_k,\textbf{m}|\textbf{Z}_{0:k},\textbf{U}_{0:k},\textbf{x}_0)=\frac{P(\textbf{z}_k|\textbf{x}_k,\textbf{m})P(\textbf{x}_k,\textbf{m}|\textbf{Z}_{0:k-1},\textbf{U}_{0:k},\textbf{x}_0)}{P(\textbf{z}_k|\textbf{Z}_{0:k-1},\textbf{U}_{0:k})}
\end{equation}

\subsubsection{Struktura}
Model pozorování udává závislost polohy vozidla a pozice landmarků, sdružená posteriorní pravděpodobnost nemůže být rozdělena na
\begin{equation}
	P(\textbf{x}_k,\textbf{m}|\textbf{z}_k)\neq P(\textbf{x}_k|\textbf{z}_k)P(\textbf{m}|\textbf{z}_k),
\end{equation}

neboť by to vedlo k chybným odhadům. Dalším zdrojem chyb je špatný odhad pozice robota. Landmarky jsou ale silně korelované, takže chybný odhad landmarku vůči mapě nevede k chybné poloze dvou landmarků navzájem.\\
\indent Velmi důležitým poznatkem bylo zjištění, že korelace mezi landmarky monotónně vzrůstá s počtem jejich pozorování (potvrzeno pouze pro lineární Gaussovský případ). Odhad pozice landmarku je tedy s narůstajícím počtem pozorování monotónně přesnější. Tento jev nastává díky, v podstatě, skoro nezávislému měření relativních pozic mezi landmarky. Ty jsou zcela nezávislé na natočení vozidla a úspěšné pozorovnání z tohoto bodu může nést další nezávislá měření relativních rozložení landmarků. \\
\indent Pohybem robota v prostoru, získává pozorováním novou pozici známých landmarků vůči sobě a dle této informace aktualizuje jejich pozici a též svoji odhadovanou polohu. Pokud již nějaký landmark není pozorován, tak je jeho pozice aktualizována dle změny landmarků pozorovaných. Při pozorování nových landmarků dochází ke korelaci s již známými, čímž se vytváří síť. Čím častěji jsou dva landmarky pozorovány při jednom měření, tím je síla korelace větší. Opětovným projížděním prostředím tak získáváme přesnější a robustnější mapu.\\


\subsection{Řešení problému SLAM}
Při řešení je potřeba adekvátně obsáhnout jak složku modelace prostředí, tak i tvorbu pohybového modelu.
Máme řadu možností jak tento problém řešit, například princip Monte Carlo, kdy rozdělujeme hustoty pravděpodobnosti odhadu pozice robota. Další možností je Markovova lokalizace, jedná se o pravděpodobnostní formu SLAM. Nejčastěji se setkáváme s reprezentací problému ve formě stavového modelu zatíženého šumem, což vede k použití rozšířeného Kalmanova filtru (v originále $extended$ $Kalman$ $filter\rightarrow EKF$). Jinou možností je ještě rozčlenit pohybový model vozidla na vzorky s obecnějším negausovským rozdělením pravděpodobnosti, v tomto případě mluvíme o použití Rao-Blackwellizedova částicového filtru. 

\subsection{EKF-SLAM}
Pohybu robota je zde popsán rovnicí
\begin{equation}
	P(\textbf{x}_k|\textbf{x}_{k-1},\textbf{u}_k)\leftrightarrow \textbf{x}_k=\textbf{f}(\textbf{x}_{k-1},\textbf{u}_k)+\textbf{w}_k,
\end{equation} 
kde $\textbf{f}(\textbf{x}_{k-1},\textbf{u}_k)$ je funkce modelující pohyb robota a $\textbf{w}_k$ jsou chyby měření.\\
\\
Model pozorování je vyjádřen vztahem
\begin{equation}
	P(\textbf{z}_k|\textbf{x}_k,\textbf{m})\leftrightarrow \textbf{z}_k=\textbf{h}(\textbf{x}_k,\textbf{m})+\textbf{v}_k
\end{equation}
kde funkce $\textbf{g}(.)$ popisuje geometrické vlastnosti pozorování a $\textbf{v}_k$ jsou chyby měření.\\
\\
Tyto dvě definice využijeme v metodě EKF k výpočtu průměru a kovariance sdruženého posteriorního rozložení\\
\indent průměr:
\begin{equation}
	\begin{bmatrix}
	\hat{\textbf{x}}_{k|k}\\\hat{\textbf{m}}_k
	\end{bmatrix}=E
	\begin{bmatrix}
	\textbf{x}_k |\textbf{Z}_{0:k}\\\textbf{m}\indent
	\end{bmatrix}
\end{equation}

\indent kovariance:
\begin{equation}
	\textbf{P}_{k|k}=\begin{bmatrix}
	\textbf{P}_{xx} \textbf{P}_{xm}\\\textbf{P}^T_{xm} \textbf{P}_{mm}
	\end{bmatrix}_{k|k}=E\begin{bmatrix}
	\begin{pmatrix}
	\textbf{x}_k-\hat{\textbf{x}}_k\\\textbf{m}-\hat{\textbf{m}}_k
	\end{pmatrix} \begin{pmatrix}
	\textbf{x}_k-\hat{\textbf{x}}_k\\\textbf{m}-\hat{\textbf{m}}_k
	\end{pmatrix}^T |\textbf{T}_{0:k}
	\end{bmatrix}
\end{equation}

Mezi hlavní problémy této metody se řadí konvergence, výpočetní složitost, sdružování dat a nelinerita. Prvním z nich, konvergence, se projevuje postupným přechodem determinantu kovarianční matice mapy a všech podkategorií dvojic landmarků k nule. Jednotlivé odchylky landmarků pak konvergují dle původních nepřesností z odhadu pozice robota a jeho pozorování.\\
\indent Výpočetní složitost je zde brána jako kvadraticky rostoucí s počtem zaznamenaných landmarků, neboť při každém zaznamenaném pozorování se aktualizují již uložené landmarky. Tento problém prošel vývojem a existují metody pracující v reálném čase s tisíci landmarky.\\
\indent Metoda EKF-SLAM je velmi náchylná na chybné spojení pozorování se známými landmarky. Jedná se zejména o problém s uzavřením smyčky, kdy dochází k opětovnému návratu na místo, ze kterého robot začínal nebo ve kterém se již nacházel.\\
\indent Nelinerita je posledním z významných problémů. Kvůli ní může dojít k větším nepřesnostem ve výsledku, neboť EKF-SLAM využívá lineárních modelů pro vyjádření nelineárního pohybu a modelu pozorování. Konvergence a konzistence modelu je tedy jistá pouze v lineárním případě. 

\subsubsection{Výpočetní složitost}
Neobvyklá struktura problému, sdružený stav složený z pozice robota a landmarků, je využita v řadě metod pro redukci výpočetní složitosti. Zde má pohybový model vliv pouze na stav pozice robota a model pozorování na dvojici robot-landmark. Základním rozdělením metod redukujících výpočetní složitost, je rozlišování optimálních, konzervativních a nekonzistentních metod. První typ, optimální metody, jsou založené na redukci daného výpočtu, výsledkem jsou pak odhady a kovariance, stejně tak, jako je tomu v případě plnohodnotné formy SLAM. U metod konzervativních dochází k odhadům s vyšší neurčitostí nebo kovariancí, většinou ale, i přes větší nepřesnost, jsou implementovány v reálném použití. Poslední možností jsou nekonzistentní metody a jedná se o algorithmy, které mají nižsí neurčitost nebo kovarianci, než algoritmy optimální. Pro řešení SLAM v praxi se ale nepoužívají.\\
\indent Prvním přístupem pro redukci výpočetní složitosti, je omezení požadovaného výpočtu rovnicí aktualizace pozorování. Výpočet časové aktualizace může být omezen metodami využívající rozšířený stav, výpočet stavu pozorování pak metodami oddělujícími rovnice dané aktualizace a obě tyto omezení vedou k redukci výpočtů, typicky jsou to optimální algoritmy. Další možností, je reformulace stavového prostoru do informační podoby, která umožnuje rozdělení výsledné matice s informacemi pro snížení výpočtů, což bývají algoritmy konzervativní. Obvykle je díky nim znatelně redukována výpočetní složitost a stále je zachována dostatečně dobrá odhadovací schopnost. Posledním přístupem, o kterém se zmíním, je submapping. Jde o rozdělování mapy na regiony, kdy následné aktualizace se týkají pouze dané oblasti a s určitou periodou poté i v rámci celé mapy. 

\subsubsection{Oddělné aktualizace}
Jedná se o metody vytvářející optimální odhady. Při implementaci základní podoby aktualizace pozorování, se při každém novém měření aktualizuje jak stav vozidla, tak i mapy. To vede ke kvadratickému nárůstu složitosti s množstvím landmarků. V této metodě se však mapa rozdělí na submapy.\\
\indent Rozlišujeme dva způsoby možné implementace. První z nich pracuje na zmenšené oblasti, ale stále si drží globální referenční souřadnice, jedná se například o algoritmus $compressed$ $EKF$ (CEFK). Druhou možností je tvorba menších map s vlastním\\ souřadnicovým rámcem neopouštějícím danou submapu, což jsou algorithmy $constrained$ $local$ $submap$ $filter$ (CLSF, v překladu - omezený lokální submapový filtr). Pokračovat budu v rozboru druhé možnosti, neboť je jednodušší a při provádění operací s velkou frekvencí opakování je méně ovlivněna linearizačními chybami. Je také stabilnější a zabraňuje příliš velkému nárůstu globální kovariance. \\
\indent Logaritmus submapy se skládá z dvou nezávislých odhadů, které si stále udržuje. Jde o vektory $x_G$ a $x_R$, kdy $x_G$ je mapa složená z globálně referencovaných landmarků a globálně referencované pozice dané submapy a $x_R$ je lokální submapa s lokálně referencovanou pozicí robota a lokálně referencovanými landmarky. Při získání pozorování se aktualizují pouze landmarky náležící aktuální submapě, ve které se robot nachází. Celkový globální odhad je pak získáván periodicky, zaevidováním submapy do mapy celé a použitím aktualizace omezení na společné vlatnosti obou map.

\subsubsection{Rozčlenění stavového prostoru}
V této metodě se vyjadřuje stavový odhad $\hat{x}_k$ a matici kovariance $P_k$ v informační formě pomocí matice informací $Y_k=P^{-1}_k$ a vektoru informací $\hat{y}_k=Y_k\hat{x}_k$. Je výhodné pro mapy s větším měřítkem, kdy spousta nediagonálních prvků bude velmi blízkých nule, což vede k možnosti nastavení těchto prvků na hodnotu nula. Může tím však vznikat malá ztráta optimality při vzniku map.\\
\indent Rozšíření stavu je rozčleňovací operace vedoucí k přesnému rozčlenění informační formy. Předpokládáme, že podmnožina stavů $x_i$ obsahuje většinu stavů mapy a po rozčlenění dosahuje pouze konstantní složitosti v čase. Můžeme tedy získávat přesné řešení díky rozšiřování stavu novým odhadem pozice robota v každém kroce a zachovat všechny předchozí pozice. Nenulové nediagonální prvky jsou pouze ty, které jsou spojené napřímo s měřenými daty.\\
\indent Dále sem musíme zahrnout marginalizaci, jež je nezbytná pro odstranění předchozích stavů pozice. Máme možnost marginalizovat všechny předchozí stavy, což vede na zhuštěnou matici informací, což je nežádoucí. Správnou volbou ukotvení pozice můžeme marginalizovat velkou čast pozic, aniž bychom vyvolaly nadměrnou hustotu matice informací.


\subsection{Rao-Blackweillizedův částicový filtr (RBPF)}
Forma SLAM založená na Rao-Blackellizedově filtru, jinak nazývaná také jako FastSLAM, je na bázi rekurzivního Monte Carlo modelu a dokáže reprezentovat nelineání stavový model. Je výpočetně nemožné aplikovat na stavový prostor s vysokým počtem dimenzí částicové filtry, je však možné redukovat velikost vzorků. \\
Sdružený stav může být jako faktor komponent vozidla a podmíněných komponent mapy
\begin{equation}
	P(X_{0:k},m|Z_{0:k},U_{0:k},x_0)=P(m|X_{0:k},Z_{0:k})P(X_{0:k}|Z_{0:k},U_{0:k},x_0)
\end{equation}
Rozdělení pravděpodobnosti zde není na jednotivých pozicích $x_k$, ale na celou trajektorii $X_{0:k}$ a tím se stávají jednotlivé landmarky na sobě nezávislými, mapa je tedy reprezentována jako soubor nezávislých gaussiánů, což znamená lineární složitost oproti kvadratické u formy EKF. Hlavními ukazateli FastSLAMu je mapa, jež je počítána analyticky a vážené vzorky, jimiž je reprezentována trajektorie pohybu. Rekurzivní odhad je proveden partikulárním filtrem pro stav pozice a EKF pro stav mapy.\\
\indent Zpracování každého landmarku probíhá zvlášť, pozice se aktualizuje stejným způsobem jako v EKF a landmarky, které nebyly zpozorovány, zůstávají na původní pozici a neaktualizují se. Vzájemnou neprovázaností landmarků však vzniká chyba v odhadu, která s časem roste.\\
\indent V čase $k-1$, je sdružený stav reprezentovaný jako 
\begin{equation}
	\{w^{(i)}_{k-1},X^{(i)}_{0:k-1},P(m|X^{(i)}_{0:k-1},Z_{0:k-1})\}^V_i
\end{equation}
V prvním kroce je pro každou částici vypočítán návrh distribuce, jež je podmíněná svojí specifickou historií. Z ní odebereme vzorek $x_k$, který je poté sdružen k historii částice $X^{(i)}_{0:k}$. Krokem dva, dle funkce důležitosti, je stanovena váha vzorků. Třetím krokem je případné převzorkování, které se provádí různě často dle implementace. Krokem posledním je provedení EKF aktualizace pro každou již zpozorovanou částici, která je při aktuálním pozorováním zaznamenána. \\
... dále o RBPF


\subsection{Asociace dat}
Jedná se velmi důležitý problém, neboť, i když během procesu tvorby mapy dojde pouze k jedné chybné asociaci dat, může to vézt k destabilizaci odhadu mapy. Často dokonce k pádu celého algoritmu.\\
\indent Z prvu se k problému přistupovalo způsobem, kdy se každé jednotlivé zachycení landmarku porovnávalo se všemi odhady nacházejícími se v blízkém okolí. Tento individuální přístup je neproveditelný, pokud je nejistá pozice robota, tedy obvzláště v málo zaplněných prostředích. \\
\indent Při popisu vzhledu je vidění jedním z hlavních způsobů snímání okolí. Podle typu senzoru se zaznamenává například tvar, barva, struktura, a tím je možné rozlišovat různé balíčky dat. To je úpté využito pro přepověd dané asociace, nejčastěji pro problém s uzavřením smyčky. Pokrok této metody přišel s počítáním metriky podobnosti přes sekvenci obrazů, místo původního jednoho. 

\subsubsection{Multihypoziční asociace dat}
Jedná se o metodu nepostradatelnou pro robustní sběr dat v přeplněném prostředí, kdy se vytváří oddělené odhady trasy pohybu pro každou asociační hypotézu. Tato funkce je však silně limitována dotupným výpočetním výkonem. Dále je metoda vyžívána při implementaci robustního SLAMu ve velkých prostředích, kdy je při uzavírání smyčky vytvořena hypotéza pro smyčku uzavřenou i pro stále neuzavřenou. Tím se bere v potaz, že je prostředí pouze podobné. 


\subsection{Reprezentace prostředí}
Původně se svět modeloval jen jako soubor landmarků majících svůj určitý tvar, později však, zejména ve venkovním, podvodním a podzemním použití, se ukázala tato metoda jako nevyhovující.\\
Grid-mapy + mapa z landmarků



\newpage

\section{GMapping}

\subsection{Úvod}
Jako velice podstatný úkol mobilního robota, bereme schopnost tvorby mapy. Ta může být kvalitně vytvářena, pokud existuje dobrý odhad pozice a pozice zase správně získávána, pokud je dostatečně kvalitní mapa. Efektivním řešením je tedy Rao-Blackwellizedův částicový filtr, který se může ještě vylepšit. Jednou možností, je zahrnout přesnost měření do návrhového rozložení, tím je získáno přesné vykreslení částic. Druhou možností, je volba vzorkovací techniky udržující rozumný počet částic. Udržuje se tak přesná mapa a snižuje se riziko vyčerpání částice, což je problém vznikající při převzorkování.\\
\indent Návrhové rozložení je získáváno vyhodnocováním pravděpodobnosti pozice robota z kombinace informací ze senzoru a odometrie. Poslední pozorování je využito pro tvorbu nových částic, odhad stavu se tedy provádí na základě více informací, než jen z odometrie. 

\subsection{Mapování pomocí RBPF}
Základním úkolem je odhad posteriorní pravděpodobnosti a tím zisk mapy a trajektorie pohybu. Odhad je prováděn díky pozorování a informaci z odometrie. Nejprve dochází k odhadu mapy, která je utvářena z pozorování, poté až trajektorie. Ta je získána z částic, které měly v rozhodovací době největší pravděpodobnost. Každá částice tedy reprezentuje část trajektorie.\\
\indent Pro výběr správné částice je použit částicový filtr, v tomto případěm typ SIR (Sampling Importance Resampling), který při mapování postupně zpracovává data ze senzoru, poté odometrii a aktualizuje sadu vzorků reprezentující posteriorní pravděpodobnost zahrnující informace o mapě a trajektorii. \\
\indent Celý proces začíná ziskem nových částic odběrem vzorků z předchozí generace návrhového rozložení. Dále je částicím nastavena vážená důležitost, aby cílové rozložení nebylo rovno tomu navrhovanému. Poté dochází k převzorkování, kdy jsou částice přepisovány úměrně jejich váženým důležitostem. Nakonec se odhaduje podoba mapy, kdy je pro každou částici, na základě trajektorie vzorku a historie pozorování, mapa vypočítána.

\subsection{RBPF s vylepšenými návrhy a adaptivním převzorkováním}
Pro zisk nové generace částic je třeba vykreslení vzorků z návrhového rozložení, kde platí úměra, čím lepší návrh, tím lepší výsledek. Kdyby byl návrh rovný cílovému rozložení, částice by měly stejnou váženou důležitost a nebylo by třeba převzorkování. \\
\indent Typicky návrhové rozložení odpovídá odometrickému pohybovému modelu, který však není optimální a to zejména, pokud je senzor výrazně přesnější než odhad pozice. Využívá se také vyhlazování pravděpodobnostní funkce, což zabraňuje částicím v okolí významné oblasti, přílišnému poklesu vážených důležitostí. Následkem je ale zkreslení mapy. To se však dá vyřešit zahrnutím posledního pozorování do generování nových vzorků. Díky tomu dochází k zaměření se na vzorkování ve významné oblasti pravděpodobnosti pozorování.\\
\indent Zlepšením návrhu se objevuje možnost získávat pro každou částici zvlášť její parametry Gaussiánského návrhu a snižuje se také neurčitost výsledných hustot pravěpodobností. Porovnávač pozorování určuje režim výnamné oblasti pravděpodobnostní funkce pozorování. Také má většinou funkci maximalizace pravděpodobnosti pozorování, tvorby mapy a počáteční odhad pozice robota. Pokud je pravděpodobnostní funkce vícerežimová, například při uzavírání smyčky, porovnávač vrací pro každou částici nejbližší maximum, což může způsobit vynechání některých maxim v pavděpodobnostní funkci.\\
\indent Převzorkování je velice důležitým aspektem určujícím výkon částicového filtru. Dochází k nahrazování vzorků s nízkou váhou, těmi s váhou vysokou. Je to nezbytný proces, neboť je potřeba konečného počtu částic pro aproximaci cílového rozložení. Může však odtranit dobré vzorky a ochudit tak částice, je proto důležité mít pro převzorkování vhodné rozhodovací kritérium a provádět ho ve správný čas. 
\begin{equation}
	N_{eff}=\frac{1}{\varSigma_{i-1}^N(w^{(i)})^2}
\end{equation} 
$w^{(i)}$ je zde normalizovaná váha částice $i$ a $N_{eff}$ jsou vzorky z cílového rozložení, které mají stejné váhové důležitosti. Pokud dochází ke zhoršení aproximace cílového rozložení, nastává větší odchylka vážených důležitostí.\\
Základní nastavení převzorkování je následující:
\begin{equation}
	N_{eff} < N/2,
\end{equation}
kde $N$ je počet částic. Tím dochází k výrazné redukci možnosti nahrazení dobrých částic a počtu převzorkování, které se vykonává pouze pokud je potřeba.\\
\indent Algoritmus probíhá následovně. Dojde k zisku odhadu pozice, který je reprezentovaný danou částicí. Je získán z předchozí pozice částice a odometrického měření od poslední aktualizace. Na základě mapy je provedeno porovnání pozorování z místa úvodního odhadu pozice, kdy se vyhledává pouze v okolí tohoto bodu. V případě selhání se pozice a váhy počítají dle pohybového modelu a následující dva kroky jsou přeskočeny. Prvním z nich vybrání sady vzorků v okolí dané pozice, vypočítání průměrů a kovarianční matice návrhu bodovým hodnocením cílového rozložení v pozici vzorku. Druhým, potenciálně přeskočeným krokem, je zakreslení nové pozice částice z Gaussovské aproximace podle zlepšeného návrhového rozložení. Dále, a to již vždy, dojde k aktualizaci vážených důležitostí a podle zakreslené pozice a pozorování je aktualizována i mapa částice. 
 
\newpage

\section{Catographer}

\subsection{Úvod}
Systém Cartographer je vyvíjen společností Google. Je vhodný pro tvorbu rozsáhlých map a zisk optimalizovaných výsledků v reálném čase. Při porovnávání snímků typem scan-to-scan dochází rychlému hromadění globálních chyb, v případě porovnávání stylem scan-to-map dochází, při správném odhadu pozice a kvalitnímu snímku z LIDARu, k velké redukci chyb, což vede k větší efektivnosti a robustnosti algoritmu. Dále se využívá porovnávání přesnosti pixelů. Jde o metodu redukující hromadění lokálních chyb a je vhodná při řešení problému uzavírání smyčky. \\

\subsection{Přehled}
Cartographer vytváří v reálném čase 2D mřížkovou mapu (grid mapu) s rozlišením na 5 cm. Submapy se vkládají na odhadované místo a daná submapa se porovnává vůči poslední zaznamenané, dochází tedy k hromadění globální chyby z odhadu pozice. Systém neobsahuje žádný částicový filtr a to z důvodu snížení hardwarových nároků na běh algoritmu.\\
\indent Submapa se po pořízení již dále nijak nepřepisuje a je zařazena mezi ostatní k porovnávání na uzavření smyčky. Pokud dojde k blízkému odhadu pozice a zároveň dostatečné shodě snímků, přidá se omezení uzavření smyčky do optimalizačního problému, tím je odhad pozice. Odhad se aktualizuje po několika vteřinách a uzavření smyčky je tedy okamžitě viditelné. 

\subsection{Lokální 2D SLAM}
Rozlišujeme dva možné přístupy, lokální a globální, v obou případech se jedná o optimalizaci pozice, přičemž pozice se zkládá z hodnoty na ose $x$, z hodnoty na ose $y$ a z natočení robota. V systému je také obsažená jednotka IMU (inertial measurment unit), která slouží k odhadu směru gravitace, což má využití při pohybu na nerovné ploše.\\
\indent Skeny prostředí se iterativně zarovnávají se snímky již obsaženými v submapě. Submapa je tedy v postatě zachycení kousku světa, který je složený z pár skenů. Lokální chyba, vznikající při její tvorbě, je poté odstraněna v globálním přístupu. Pro každý bod mřížky je definován odpovídající pixel skládající se ze všech pixelů nejblíže danému bodu.\\
\indent Při přidání skenu do pravděpodobnostní mřížky je počítána množina zasažených bodů mřížky a bodů minutých. Při zásahu se nejbližší bod mřížky vloží do množiny zásahů. Při minutí je vložen bod mřížky sdružený se všemi pixely, které jsou protínány jedním paprskem mezi počátkem skenování a každým snímacím bodem. Nepřidávají se sem body již přidané do množiny zásahů. Doposud nepozorované body mřížky mají přiřazenou pravděpodobnost minutí či zásahu, podle toho, jestli se v jedné z těchto množin vyskytují. Již pozorovaným bodům se pak aktualizují pravděpodobnosti minutí a zásahu.\\
\indent Před vložením skenu do submapy se ještě využívá Ceres scan matching, který optimalizuje pozici skenu vůči submapě. Jedná se o maximalizaci pravděpodobnosti výskytu v dané oblasti.  

\begin{equation}
	argmin_\xi\Sigma^K_{k=1}(1-M(T_{\xi}h_k))
\end{equation}

$H$ je zde informace o bodech skenu, $M$ je pravděpodobnostní mřížka, $\xi$ je pozice snímání skenu a $T_{\xi}$ je pozice skenu vůči submapě. Dochází k transormaci, kdy body skenu se transformují do submapy.

\subsection{Uzavírání smyčky}
Daný systém pracuje v oblasti submap s porovnáváním scan-to-scan, hromadí se v něm tedy lokální chyby, ale pár snímků za sebou má vůči sobě chybu minimální. Relativní pozice skenů se ukládají a v případě, že se submapa nezmění, všechny další páry ze skenů a submapy se předkládají k porovnávání pro uzavření smyčky. To vše běží na pozadí a pokud je nalezena shoda, dojde k uložení relativní pozice mezi optimalizační problémy.\\
\indent Optimalizační problém je problém nelineárních nejmenších čtverců. Díky tomu můžeme jednoduše přidávat zbytky pro zohledňování dalších dat. Jednou za pár sekund je, pro optimalizaci pozice skenu vůči daným omezením, spuštěn Ceres scan matcher. Omezeními jsou myšlena relativní pozice $\xi_{ij}$ a kovarianční matice $\Sigma_{ij}$.

\begin{equation}
	argmin_{\Xi^m,\Xi^s}\frac{1}{2}\Sigma_{ij}\rho(E^2(\xi_i^m,\xi_j^s,\Sigma_{ij},\xi_{ij})),
\end{equation}\\
kde $\rho$ je ztrátová funkce, například tedy Hubertova ztráta. Jedná se o snížení vlivu odlehlých hodnot, které přidávají nesprávná omezení do optimalizačního problému.

\indent Při uzavírání smyčky se vyžívá také branch-and-bound scan matching, což se dá přeložit jako porovnávání větví a mezí. Zde se metoda zaměřuje na přesnou shodu pixelů. Podmnožiny možností jsou popsány jako uzly stromu, kdy kořenový uzel obsahuje všechna možné možnosti ($W$). Potomci uzlu dohromady utváří stejný soubor možností, jako samotný rodičovský uzel. Listy jsou brány jako singlety, odpovídají jedinému proveditelnému řešení. Tento přístup dává stejné řešení jako ten předchozí, dokud je hodnota $score(c)$ vnitřních uzlů horní mezní skóre jeho prvků, neexistuje tedy žádné řešení, než to doposud známé.

\begin{equation}
	\xi^*=argmax_{(\xi\epsilon W)}\Sigma_{k=1}^KM_{nearest}(T_{\xi}h_k)
\end{equation}\\
$W$ je zde vyhledávací okno a $M_{nearest}$ je rozšíření pravděpodobnostní mřížky $M$ na všechny $R^2$ zaokrouhlením argumentů do nejbližšího bodu mřížky. Rozšířená hodnota bodu mřížky ukazuje na odpovídající pixel.\\
\indent Výběr uzlů probíhá prohledáváním do hloubky. Efektivnost algoritmu hodně závisí na podobě stromu, zda-li má pro výpočet dobrou horní mez a dobré aktuální řešení. Důležitým termínem je zde prahová hodnota skóre. Jedná se o hodnotu, pod níž není v zájmu jít a řešení je tedy nevyhovující. Díky tomu se nepřidávají špatné shody jako omezení pro uzavírání smyčky. Pro rychlost algoritmu je důležité rozhodování o průchodu stromem. Pro každého potomka je vypočítána horní hranice skóre a je vybrán ten nejslibnější, což je uzel s největším mezním počtem.\\
\indent Každý z uzlů je popsán pomocí tuple integerů: 
\begin{equation}
	c=(c_x,c_y,c_\Theta,c_h)\epsilon Z^4,
\end{equation}\\
kde $c_h$ je výška uzlu. Pokud $c_h=0$, uzel je list.\\
\indent Horní meze jsou pak počítány na vnitřních uzlech, zde je zájem o výpočetní úsilí a kvalitu spojení. 
\begin{equation}
	score(c)=\Sigma_{k=1}^Kmax_{(j\epsilon \bar{W})}M_{nearest}(T_{\xi_j}h_k)
\end{equation}

\newpage

\section{Hector SLAM}

\subsection{Úvod}
Hlavním znakem systému Hector SLAM, je jeho rychlost a nízké výpočetní nároky oproti předchozím dvěma typům, které jsou v této práci rozebírány. Je tedy vhodný pro implementaci v menších autonomních systémech, pro rychlý pohyb terénem a nehodí se pro uzavírání velkých smyček.\\
\indent Vnitřní skladba systému je ze tří hlavních částí, a to 2D SLAM, běžící jako soft real time, 3D navigace, která je hard real time a jednotka IMU, která se nachází i v systému Cartographer. Hector SLAM je braný jako front-end SLAM, což znamená, že dochází k odhadu stavu robota v reálném čase. Back-end SLAM optimalizuje poziční graf vzhledem k omezením mezi pozicemi.

\subsection{Přehled}
Daný systém se musí převézt z 3DOF pomocí naklonění a rotace na 6DOF, kdy navigační filtr spojí měření z inerciální jednotky (IMU) a dalších senzorů. Tím dojde k získání 3D řešení a díky 2D SLAMu je obsažena informace o poloze v prostoru. Oba odhady se aktualizují nezávisle na sobě a jsou propojeny jen velmi málo.\\
\indent Kvůli znatelnému posunu odhadů integrované pozice a rychlosti, který je způsoben šumem v senzorech, zahrnujeme do systému další informace. Jedná se například o porovnávání snímků, snímání magnetického pole, senzor barometrického tlaku a nebo měření rychlosti kol.

Pohyb robota je popsán:
\begin{equation}
	\dot{\Omega}=E_\Omega.\omega
\end{equation}
\begin{equation}
	\dot{p}=v\\
\end{equation}
\begin{equation}
	\dot{v}=R_\Omega.a+g
\end{equation}\\
$\Omega=(\phi,\theta,\psi)^T$ je zde informace o otáčení, stoupání a natočení. Vektor $x=(\Omega^T p^T v^T)^T$ reprezentuje 3D stav, $p,v$ jsou pozice a rychlost platformy v navigačním rámci. Vektor $u=(\omega^T a^T)^T$ je vstupní vektor pro inerciální měření, $\omega=(\omega_x,\omega_y,\omega_z)^T$ je úhlová rychlost a $a=(a_x,a_y,a_z)^T$ je zrychlení. $R_\Omega$ je pak matice směrových cosinů, $E_\Omega$ je mapování natočení těla na deriváty Eulerova úhlu a $g$ je vektor gravitace.\\

\subsection{2D SLAM}
Jako reprezentace prostředí je zde vybrána osvědčená metoda mřížkových map. Odhadovaná pozice robota slouží pro transformaci skenu na lokální stabilizovaný souřadnicový rámec. Odhadovanou orientací a přidruženými hodnotami je ze skenu vytvořen oblak bodů, který je pro odstranění odlehlých bodů předzpracováván. Filtrace probíhá pouze na základě souřadnic koncového bodu, kdy jsou pro porovnávání skenu použity pouze koncové body v rámci skenovací roviny.\\
\indent Jak již bylo poznamenáno, je zde použita struktura mřížkových map, to vede k omezení přesnosti a neumožnění přímého výpočtu interpolovaných hodnot a derivátů. Pro oba odhady se tedy využívá interpolační schéma, díky kterému ta možnost je. Jakoukoliv souřadnici, kterou potřebujeme nahradit, aproximujeme pomocí čtyř nejbližších integerových souřadnic. \\
\indent Laserové skenery jsou již velice přesná zařízení, jejich měření zatěžuje minimální šum a snímky se dají vytvářet s vysokou frekvencí. Měření pomocí laseru je tedy mnohem přesnější než to odometrické, což je jeden z důvodu, proč je odometrie v tomto systému zcela vynechána. Při zarovnávání snímků s již známou mapou není potřeba hledat žádná spojení mezi koncovými body, ale dochází k porovnávání s předchozími skeny.  

Hledání transformace při nejlepším sladění skenu s mapou:
\begin{equation}
	\xi^*=argmin_\xi\Sigma_{i=1}^n[1-M(S_i(\xi))]^2
\end{equation}
$M(P_m)$ je zde hodnota obsazenosti a $S_i(\xi)$ jsou souřadnice koncového bodu $s_i=(s_{i,x},s_{i,y})^T$, přičemž $\xi$ jsou souřadnice robota.\\
 
Optimalizace chyby měření: 
\begin{equation}
	\Sigma_{i=1}^n[1-M(S_i(\xi+\Delta\xi))]^2\rightarrow 0,
\end{equation}
kde $\Delta\xi$ je odhad $\xi$.\\
\indent Při tvorbě mapy dochází často k nalezení pouze lokálního minima, reprezentací mapy pomocí více rozlišení se tak toto riziko znatelně redukuje. Utváří se mřížkové mapy, kdy každá další má poloviční rozlišení, než ta předchozí. Je tak uloženo více map, které se součastně aktualizují podle odhadu aktuální pozice, který je generovaný zarovnávacím procesem. Díky tomuto přístupu jsou mapy konzistentní a nepotřebují převzorkovávání. Zarovnávání skenů probíhá pouze na mapě s nejvyšším rozlišením a tento odhad je pak, jak je uvedeno, použit pro ostatní mapy. Mapy s nízkým rozlišením jsou tedy v podstatě dostupné okamžitě, a lze je hned použít pro odhad trasy. 

\subsection{Odhad 3D stavu}
Pro odhad úplného 6D stavu (3D stav + informace o naklonění z gyroskopů a informace z akcelerometrů) slouží navigační filtr běžící v reálném čase. Filtr je asynchronně aktualizován při příchodu odhadované pozice z porovnávače nebo jiných informací ze senzorů. Filtr je implementován ve formě EKF, jedná se tedy o nelineární filtr a jako jeho známé vstupy se berou inerciální měření. Rychlost a pozici aktualizujeme integrací zrychlení. Aby se ale zabránilo nezávislému růstu odhadu stavu při nepřítomnosti měření, dochází k aktualizace pseudo-nulové rychlosti, které se spouští při dosažení odchylky určité prahové hodnoty a zajišťuje tak stabilitu systému.\\
\indent Při integraci systému jsou přítomny základní dva celky, 2D SLAM a 3D stavový odhad, které spolu musejí komunikovat v obou směrech. Jejich fungování neni nijak synchronizováno, odhad stavu běží typicky s vyšší frekvencí. Odhadovaná pozice z EKF je brána jako počáteční odhad pro optimalizaci porovnávání. Opačným směrem je použit pro spojení pozice ze SLAM s celkovým odhadovaným stavem kovarianční průsečík (CI - covariance interseption).

\end{document}